\documentclass{article}

\title{A game theoretic model of the behavioural gaming that takes place at the EMS
 - ED interface}
\author{Michalis Panayides}

\begin{document}

\maketitle
\begin{abstract}
    Operational Research (OR) techniques provide a toolkit of mathematical 
    modelling approaches that are routinely used for problem solving in many
    sectors. Recent research shows that behavioural studies may accompany such
    techniques and offer some additional insight to the problem studied and 
    thus enrich some of the already existing traditional methods. Ethnography is 
    considered to be one of the main such approaches.

    Ethnographic studies have been around since the \(19^{th}\) century and
    their main guiding methodology is that a researcher that aims to study 
    a specific social setting should acquire a deeper understanding of the 
    cultural norms and values. The main technique that accompanies ethnography 
    is participant observation, where the researcher participates in the social 
    setting they wish to study recording their observations.

    The application of these ideas and principles considered here is the emergent 
    behaviour that takes place at the interface between Emergency Medical Services 
    (EMS) and the Emergency Department (ED). Numerous decisions are taken by both 
    patients and staff alike that determine the level of workflow and the patient 
    pathway. There is empirical evidence to suggest that imposing targets in the ED 
    results in gaming at the interface of care between the EMS and ED. Multiple 
    scenarios are examined where an ambulance service needs to distribute patients 
    between neighbouring hospitals. The interaction between the hospitals and the 
    ambulance service is modelled in a game theoretic framework, supported by a
    novel Markov model, where the ambulance service has to decide how many patients 
    to distribute to each hospital.

    This work brings together ethnography and game theory to understand the emergent
    behaviour in healthcare settings. The talk will present the problem being tackled
    at the interface of the EMS and the ED that is to be further informed by ethnographic methods.
        
\end{abstract}
    
\end{document}
